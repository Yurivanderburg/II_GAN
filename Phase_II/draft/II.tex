\section{Intensity Interferometry with ICTA}
R. Hanbury Brown and Robert Q. Twiss first carried out II. They successfully installed the Narrabri Stellar Intensity Interferometer to resolve the diameter of bright stars at visible wavelengths \citep{brown1954lxxiv, brown1974intensity}.
\subsection{The signal for II}\label{sec:signal}
In a simple case, an II consists of two light collectors, for example, MAGIC-I and MAGIC-II, which are pointed to a star and measure the intensity $I_1(t), I_2(t)$ respectively \citep{acciari2020optical, dravins2013optical}. The signals from both detectors are cross-correlated and averaged over time, which results in an equation form
\begin{equation}
	g^{(2)} = \frac{\left\langle I_1(t) \cdot I_2(t + \tau) \right\rangle}{\langle I_1(t) \rangle \cdot \langle I_2(t) \rangle} 
	\label{eqn:HBT}
\end{equation}
where $\tau$ is the time delay between both telescopes. If the light source is chaotic and randomly polarized, we have:
\begin{equation}
	g^{(2)} = 1 + \frac{\Delta f}{\Delta \nu} \abs{V_{12}}^2
	\label{eqn:HBT2}
\end{equation}
here, $\Delta f$ is the electronic bandwidth, $\Delta \nu$ is the optical bandwidth, and $\Delta f \ll \Delta \nu$ \cite{acciari2020optical}. $V_{12}$ is the Fourier transform of the source brightness distribution and is called complex visibility. Eqn.~\ref{eqn:HBT2} carry the information about the source's structure and shape. However, the phase information is lost due to the involvement of the absolute value of $V_{12}$. In observational astronomy, the correlation is often expressed in terms of the normalized contrast, given by:
\begin{equation}
	c = \frac{\left\langle \left( I_1(t) - \left\langle I_1 \right\rangle \right) \cdot \left( I_2(t + \tau) - \left\langle I_2 \right\rangle \right) \right\rangle}{\langle I_1(t) \rangle \cdot \langle I_2(t) \rangle} = g^{(2)} - 1
\end{equation}
where, $\left\langle I_1 \right\rangle$ and $\left\langle I_2 \right\rangle$ are the mean intensity from two telescopes.

\subsection{The Signal-to-Noise Ratio for II}
\begin{figure}
	\centering
	\includegraphics[width=\linewidth]{fig/telescope.png}
	\caption{The telescope configuration with similar properties each used to simulate the signal for II.}
	\label{fig:teles}
\end{figure}
\begin{figure}
	\centering
	\includegraphics[width=\linewidth]{fig/baseline.png}
	\caption{The tracking of baselines with four telescopes arranged in fig.~\ref{fig:teles} for one night of observation.}
	\label{fig:base}
\end{figure}
\begin{figure}
	\centering
	\includegraphics[width=0.8\linewidth]{fig/ellipse/ellipse6018.jpg}
	\caption{This figure shows the simulated fast rotating star. The brightness is highest at the pole and there is gravitational darkening at the equator.}
	\label{fig:image}
\end{figure}
\begin{figure*}
	\centering
	\begin{subfigure}{0.5\linewidth}
		\includegraphics[width=\linewidth]{fig/ft/ft.jpg}
		\caption{The fourier transform of source.}
	\end{subfigure}\hfill
	\begin{subfigure}{0.5\linewidth}
		\includegraphics[width=\linewidth]{fig/ft/ft_log.jpg}
		\caption{The logarithmic fourier transform of source.}
	\end{subfigure}
	\caption{Absolute value of the two-dimensional Fast Fourier Transform of fig.~\ref{fig:image} measured by Intensity Interferometry, and observation of maximum (u, v) plane with finite number of baselines can be completed with help of earth's rotation.}
	\label{fig:ft}
\end{figure*}
\begin{figure*}
	\centering
	\begin{subfigure}{0.5\linewidth}
		\includegraphics[width=\linewidth]{fig/ft/ft_base.jpg}
		\caption{The fourier transform with baselines.}
	\end{subfigure}\hfill
	\begin{subfigure}{0.5\linewidth}
		\includegraphics[width=\linewidth]{fig/ft/ft_log_base.jpg}
		\caption{The logarithmic fourier transform with baselines.}
	\end{subfigure}
	\caption{The left panel shows the absolute value of the two-dimensional Fast Fourier Transform of fig.~\ref{fig:image} measured by baselines shown in fig.~\ref{fig:teles}. The right panel shows the same on logarithmic scale}
	\label{fig:ft_base}
\end{figure*}
The main physics channel of ICTAs is the study of Very High Energy gamma rays ($\geq 30$ GeV) originating from particle showers in the atmosphere. Telescopes have an array of mirrors, which focus the light onto an array of photo-multiplier tubes (PMTs) \cite{aleksic2016major}. Here, fig.~\ref{fig:teles} shows an arrangement of four IACTs with similar properties each. For II, the light of a stellar source is focused on a single PMT, which has a filter in front. The purpose of this filter is to efficiently transmit light around mean observational frequency, which protects the PMTs from excessive light. It is necessary, as II observations are mainly done during full moon nights, where it is too bright for Very High Energy $\gamma$-ray observations. The optical signal of the PMTs is transmitted with optical fibers, converted to an electrical signal, and digitized \cite{acciari2020optical}. The measurable observable is the Pearson's correlation coefficient:
\begin{equation}
	\rho(\tau) = \frac{\left\langle \left( I_1(t) - \left\langle I_1 \right\rangle \right) \cdot \left( I_2(t + \tau) - \left\langle I_2 \right\rangle \right) \right\rangle}{\sqrt{\langle \left( I_1(t) - \langle I_1 \rangle \right)^2 \rangle} \cdot \sqrt{\langle \left( I_2(t) - \langle I_2 \rangle \right) ^2 \rangle}}
	\label{eqn:pearson}
\end{equation}
Since $I_1$ and $I_2$ are proportional to the direct current of the PMTs ($DC_i$), the normalized contrast can be calculated as in equation \ref{eq:norm_contrast2}. One must also correct for a non-constant gain of the PMTs, as well as the moon. The latter can be done by measuring the background light with an additional PMT with no mirrors focused on it \cite{acciari2020optical}. 
\begin{equation}
	c \propto \frac{\rho}{\sqrt{\left(DC_1 \cdot DC_2 \right)}}
	\label{eq:norm_contrast2}
\end{equation}
It must be noted that the calculation of Pearson's correlation coefficient is non-trivial: MAGIC makes use of the convolution theorem for discrete Fourier transforms because it is computationally more efficient.

The significance of the signal can be expressed as the signal-to-noise ratio (SNR), which depends on many factors. However, most importantly, it is proportional to the absolute value of visibility, which itself depends on the distance between the telescopes, shown in equation (\ref{eq:SNR}) \cite{acciari2020optical}.  
\begin{equation}
	SNR = A \cdot \alpha \cdot q \cdot n \cdot \abs{V_{12}}^{2}(d) \cdot \sqrt{b_v} \cdot F^{-1} \cdot \sqrt{\frac{T}{2}} \cdot \sigma
	\label{eq:SNR}
\end{equation}
here, $A$ is the total mirror area, $\alpha$ the quantum efficiency of the PMTs, $q$ the quantum efficiency of the optics, $n$ the differential photon flux from the source, and $b_v$ the cross-correlation bandwidth. The noise of the PMTs is accounted for with $F$, $T$ denotes the observation time, and $\sigma$ is the normalized spectral distribution of the light (including filters) \cite{acciari2020optical}. While most of the parameters can be optimized with hardware, the only way to obtain a better SNR is to increase the observation time with fixed telescopes.  

\subsection{Baseline considerations}
The measurement of the size of stellar objects through absolute visibility depends on the distance between the telescopes, which is called the baseline $B$. 
\begin{equation}
	V_{12}(B) = \frac{c(B)}{c(0)}
	\label{eq:angular_size_meas}
\end{equation}
However, this work needs a good SNR value for high precision of measurement, so the large covered observational plane with telescopes is a necessity for II \cite{acciari2020optical, abeysekara2020demonstration}. If the source is at the zenith, the coordinates in the Fourier plane ($u,v$) are given by:
\begin{equation}
	(u,v) = \frac{1}{\lambda} (B_N, B_E)
\end{equation}
where $B_N$ and $B_E$ are the baselines expressed in north and east coordinates. Since not all sources are at the zenith, and the telescopes are stationary, Earth's rotation plays an important role in covering the maximum observational plane through the rotated baselines. It is given by equation (\ref{eq:baseline_rot}), which traces an ellipse for every pair of telescopes. Furthermore, the different altitudes $B_A$ of the telescopes must be considered \cite{dravins2013optical, saha2020theory}.  
\begin{equation}
	\begin{pmatrix} u\\ v\\ w\\ \end{pmatrix} = R_x(\delta) \cdot R_y(h) \cdot R_x(-l) \begin{pmatrix} B_N\\ B_E\\ B_A\\ \end{pmatrix}
	\label{eq:baseline_rot}
\end{equation}
here, $\delta$ is the declination and $h$ is the hour angle of the stellar source, and $l$ is the latitude of the telescopes. The three matrices $R_i$ correspond to the fundamental representation of the SO(3) group \cite{saha2020theory}. Fig.~\ref{fig:base} shows the track of six baselines generated from the telescopes (fig.~\ref{fig:teles}) due to Earth's rotation. Since every pair of telescopes traces an ellipse in the Fourier plane, the total number of ellipses scales as follows:
\begin{equation}
	\label{eq:N_telescopes}
	\mathcal{N} = \frac{N_T \cdot (N_T -1)}{2}
\end{equation}

As the number of baselines increases non-linearly, II benefits greatly from a large number of telescopes. So, the planned Cherenkov Telescope Array (CTA) will cover the maximum observational plane and provide insight into stellar objects with optical wavelengths in the future.

\subsection{An Object: Fast Rotating Stars}
In our work, we simulate a single fast-rotating star to test image reconstruction using a GAN. Fast rotation causes stars to take on an oblate shape, flattening at the poles and bulging at the equator due to the existence of stronger centrifugal force \cite{von1924radiative, 1999A&A...347..185M}. Fig.~\ref{fig:image} shows one of the simulations of a fast-rotating star, where the star takes on an elliptical shape with brightness distributed across the object. The brightness is at its maximum at the poles and minimum at the equator, a phenomenon known as gravity darkening \cite{lucy1967gravity}. This effect has been observed first through the interferometric and spectroscopic data from the CHARA Array for the fast-rotating star Regulus \cite{mcalister2005first}. Fast-rotating stars are crucial for understanding various astrophysical processes, including stellar evolution, structure, and dynamics over time.

Intensity Interferometry observes the stellar object in the form of photons, and the correlation of it results in squared visibility (explained in subsection.\ref{sec:signal}). According to Van Cittert Zernike's theorem, this signal is the Fourier transform of brightness distribution in the sky. Fig.~\ref {fig:ft} shows the same statement for the source shown in fig.~\ref{fig:image} using II on linear and logarithmic scale. However, we do not have an existing technique to observe the complete signals, and we could capture some parts of it only with a one-night simulation, which has been shown in fig.~\ref{fig:ft_base}. It is the covered observational plane using the baselines shown in fig.~\ref{fig:teles}. However, the upcoming CTA promises to cover the maximum observational plane for stellar objects.