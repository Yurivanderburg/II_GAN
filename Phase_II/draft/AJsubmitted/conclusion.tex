\section{Conclusion}

Intensity Interferometry (II) is re-emerging as a promising technique to overcome the challenges of very long baseline interferometry in the optical wavelength range.  However, compared to radio-interferometry, optical interferometry faces an important hurdle: photon correlation captures only the magnitude of the interferometric signal, resulting in a loss of phase information.

This work addresses the challenge of phase retrieval in II using a machine-learning technique, specifically a conditional Generative Adversarial Network (cGAN). Our study demonstrates that applying a cGAN to II data successfully recovers the size, shape, and brightness distribution of a fast-rotating star. Evaluations based on image moments—specifically, the monopole, second, and third-order moments—support the effectiveness of cGAN in achieving accurate image reconstruction from a simulation of II from a single site with four telescopes.  

While the results of this study highlight the significant potential of machine learning, and in particular the applicability of cGAN, for image reconstruction in II, several aspects require further refinement. First, an important factor in the reconstruction process is the extent of Fourier plane coverage, which depends on the number of available telescopes and the total observing time. The reasonable success of this piece of work suggests that a network of higher number of telescopes providing higher number of baselines and greater coverage of the $(u,v)$ plane signal, projects of image reconstruction of more complicated stellar systems can be undertaken. Future work might explore different observatory layouts to assess their impact on image reconstruction quality.  Second, detector efficiencies, which impact the signal-to-noise ratio (SNR) of actual observational data, have not yet been incorporated; addressing these factors will be crucial for more accurate SNR estimation.  Third, exploring and comparing alternative methods for image generation could reveal approaches that outperform cGAN in reconstructing stellar images with II. Fourth, experimenting with different loss functions could provide additional insights into the reconstruction quality. Although further testing is needed to refine the GAN and enhance its robustness and reliability, our findings suggest that machine learning is a promising approach for phase reconstruction in II.
