\section{Image Reconstruction: Results and Evaluation}
                      
In this section, we examine the performance of the GAN Model whose architecture and choice of hyperparameters have been discussed above. We subject the trained Model to the task of phase retrieval and image reconstruction on the Testing Dataset.

\subsection{Visual Evaluation of Images Predicted by the Model}
Fig.~\ref{fig:six_GAN} and Fig.~\ref{fig:nine_GAN} demonstrate the success of the trained Model in reconstructing the images of a sample of the fictitious fast-rotators drawn from the Testing Set.

The four panels from left to right in each row of Fig.~\ref{fig:six_GAN} and Fig.~\ref{fig:nine_GAN} represent the following:
\begin{itemize}
\item{The left panel represents the sparse II pattern obtained by the simulated observation using the four IACTs illustrated in Fig.\ref{fig:teles}. As explained earlier, this image acts as part of the input, namely its ``condition'' part. This image acts as the condition that the sky-images of the star generated by the Generator must conform to.}
 
\item{The second panel from left displays the real image, or ground truth target image, which the Discriminator loss function uses to distinguish from the images generated by the Generator.}

\item{The third panel from left presents the reconstructed (or predicted) image corresponding to the ground truth (second panel) and generated by the trained Model. The similarity of these two images indicates the success of the Model in its stated objective of image reconstruction.  We remark that rotating the images by $180^\circ$ would not change the data.  That is, each predicted image contains an arbitrary choice among two possible orientations, differing by $180^\circ$.}

\item{The right panel shows the residuals between the ground truth target image and the predicted image in the interferometric plane. The small values are indicative of the success of training the Model.}
\end{itemize}

The predicted images and the residuals presented in the third and the fourth columns (from left) of Fig.~\ref{fig:six_GAN} and Fig.~\ref{fig:nine_GAN} show encouraging results, conveying visual information about the source's size, shape, and brightness distribution across its surface fairly accurately. This has been achieved on the basis of the input provided by II observation using only 15 and 36 baselines, corresponding to six and nine telescopes, respectively.  Further improvements could surely be achieved by increasing the number of telescopes to enhance the coverage of the $(u, v)$ plane. A closer examination of this proposition might be able to contribute to the design and instrumentation aspects in the existing and upcoming CTAO.

\begin{figure*}
	\includegraphics[width=.24\linewidth]{fig/moments/mom0.png}\hfil
	\includegraphics[width=.24\linewidth]{fig/moments/mom3.png}\hfil
	\includegraphics[width=.24\linewidth]{fig/moments/mom4.png}\hfil
	\includegraphics[width=.24\linewidth]{fig/moments/mom5.png}\hfil
	\includegraphics[width=.24\linewidth]{fig/moments/mom6.png}\hfil
	\includegraphics[width=.24\linewidth]{fig/moments/mom7.png}\hfil
	\includegraphics[width=.24\linewidth]{fig/moments/mom8.png}\hfil
	\includegraphics[width=.24\linewidth]{fig/moments/mom9.png}
	\caption{The second order central moment of the brightness distribution of the images is analogous to the moment of inertia of a mass distribution. The panels of this figure show the comparison of the second order moments ($\mu_{11}$, $\mu_{20}$, and $\mu_{02}$) of the predicted images vs. the ground truth images from left to right respectively. Approximate equality of these moments is evident. The small scatter in the moments of the predicted images is indicative of a balanced training of the Model.}
	\label{fig:struc}
\end{figure*}

\subsection{Evaluation of the Model using Moments}
The reconstructed images are visually compelling, demonstrating the Model's effectiveness in using II to reconstruct images. However, visual assessment alone is insufficient; statistical evaluation of the generated images in comparison with the ground truth images is necessary to validate the results. We present here the results of our calculation and comparison of the moments of distribution pixel brightness in the target ``Ground Truth'' images and the ``Predicted'' images.

Image moments capture key properties of the reconstructed objects, such as shape, size, and intensity distribution, by quantifying features like position, orientation, and brightness distribution. By comparing the moments of the Model-generated images with those of the ground truth, we can objectively assess the consistency and accuracy of the reconstruction. This approach provides a reliable framework for evaluating reconstruction quality, as image moments can reveal subtle differences in geometric and intensity properties that might not be apparent through visual inspection alone.

The raw moment $M_{ij}$ of an image is defined as \citep{hu1962visual}
\begin{equation}
	M_{ij} = \sum_{x} \sum_{y} x^i y^j I(x, y)
	\label{eqn:Mom}
\end{equation}
where $I(x,y)$ represents the intensity at pixel $(x,y)$. The zeroth order raw moment, or monopole, represents the total intensity of an image. It is computed by summing all pixel values across the image, yielding an overall intensity measure. In this context, analyzing the monopole provides the total pixel brightness of the images of the fictitious stars. According to Eq.~\eqref{eqn:Mom}, the monopole of an image is calculated as:
\begin{equation}
	M_{00} = \sum_{x} \sum_{y} I(x, y).
\end{equation}

While the monopole effectively represents the total brightness, it does not provide information about the position, shape, size, or detailed brightness distribution of the fast-rotating stars. For these aspects, higher-order moments are necessary.

The first order raw moments normalized by the respective monopole moments of the images given by 
\begin{eqnarray}
&&x_c = \frac{M_{10}}{M_{00}} = \frac{\sum_{x,y} x \cdot I(x,y)}{\sum_{x,y} I(x,y)} \nonumber \\
&&y_c = \frac{M_{01}}{M_{00}} = \frac{\sum_{x,y} y \cdot I(x,y)}{\sum_{x,y} I(x,y)}
\end{eqnarray}
represent the centroids $(x_c, y_c)$ of the pixel brightness distribution of the images.

Furthermore, these calculated centroids are instrumental in analyzing the shape, size, and brightness distribution of the stars using higher-order image moments. To this end, the central moment of an image is calculated according to:
\begin{equation}
	\mu_{pq} = \frac{1}{M_{00}}\sum_{x} \sum_{y} (x - x_c)^p (y - y_c)^q I(x, y).
\end{equation}
The sum of \(p\) and \(q\) defines the order of the central moment. 

The second order central moment of the brightness distribution of the images is analogous to the moment of inertia of a mass distribution. 

Fig.~\ref{fig:struc} presents the comparison of monopole, second-order central moments (\(\mu_{11}, \mu_{20}, \mu_{02}\)), and third-order central moments (\(\mu_{30}, \mu_{03}, \mu_{21}, \mu_{12}\)) which are used to study the shape, structure and brightness distribution of a fast-rotating star along the line of sight, respectively using six and nine telescopes (as explained in the upcoming subsection). All these plots demonstrate an approximate equality among these moments, thereby confirming the success of applying the Model to reconstruct images with II. The small scatter in the moments of the predicted images indicates robust learning of the Model without over-fitting.

\subsection{The reconstructed Parameters for object}
The centroids \((x_c, y_c)\) indicate only the center of the star and its spatial location in the image. In contrast, the second-order central moments determine the orientation, semi-major axis, and eccentricity relative to the source's center \citep{teague1980image}. These moment-based parameters fully describe the two-dimensional ellipse that fits the image data.

The orientation of a fast-rotating star along the line of sight is defined in terms of second-order central moments as
\begin{equation}
	\theta = \tfrac{1}{2}\arctan \left(\frac{2\mu_{11}}{\mu_{20} - \mu_{02}}\right).
	\label{eqn:orn}
\end{equation}
The semi-major and semi-minor axes of the stellar object are computed using the second-order central moments and are denoted as \(a\) and \(b\), respectively.
\begin{equation}
	\begin{aligned}
		&a = 2\sqrt{mp + \delta} \\
		&b = 2\sqrt{mp - \delta}
	\end{aligned}
	\label{eqn:semi}
\end{equation}
where,
\begin{equation}
	mp = \frac{\mu_{20} + \mu_{02}}{2}
	\label{eqn:mp}
\end{equation}
and
\begin{equation}
	\delta = \frac{\sqrt{4\mu_{11}^2 + (\mu_{20} - \mu_{02})^2}}{2}.	
	\label{eqn:delta}
\end{equation}
Using the calculated axis values, the eccentricity of the fast-rotating star is determined as:
\begin{equation}
	e = \sqrt{1 - a/b}.
	\label{eqn:eccen}
\end{equation}
Eqs.~\ref{eqn:orn}-\ref{eqn:eccen} describe the elliptical nature of the stellar object (in this case, a fast-rotating star) and provide information on its shape and size, depending on the computed values. In contrast, the brightness distribution is characterized by skewness, which is quantified using third and higher-order moments.

