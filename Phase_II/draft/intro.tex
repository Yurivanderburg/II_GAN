\section{Introduction}

Intensity Interferometry (II) emerged in the 1970s as a groundbreaking technique for measuring the diameters, orbits, and limb-darkening coefficients of 32 stars in the N-body system. Despite these significant achievements, the method did not gain widespread adoption due to the then-required highly sensitive photon detectors and sophisticated data analysis techniques. However, advancements in computational methods and modern electronics, which now offer time resolution of detectors in the nanoseconds (ns) to picoseconds (ps) range, have renewed II as a feasible technique for resolving astrophysical objects at visible wavelengths. Current simulations with ns resolution of photon detectors show that II is effective in achieving high precision measurement of parameters for stellar objects \citep{10.1093/mnras/stab2391, 10.1093/mnras/stac2433}. However, a significant drawback of II is the loss of phase information. Consequently, a full reconstruction of the brightness distribution of a source with II necessitates phase information.

Several theoretical and computational approaches to phase reconstruction with II have been proposed. \cite{gamo1963triple} introduced the concept of triple-intensity correlation, which \cite{goldberger1963use} further applied in an experiment to observe scattered particles in microscopic systems. Sato conducted experiments to measure the diameter and phase of asymmetrical objects, suggesting that triple correlation could extend II to image stellar bodies \citep{sato1978imaging, sato1979computer, sato1981adaptive}. Despite its potential, the sensitivity of the signal to noise poses a significant challenge for this approach.

Later \cite{holmes2010two} proposed an alternative method, utilizing the Cauchy-Riemann relations to reconstruct 1-D images and extending this to 2-D images across a range of signal-to-noise (SNR) values. This algorithm was applied to simulated data of stellar objects using II, considering existing and forthcoming Imaging Cherenkov Telescope Arrays (IACTs) with a large number of telescopes \citep{nunez2010stellar, nunez2012high, nunez2012imaging}.

In this paper, we propose an alternative approach using Generative Adversarial Networks (GANs) \citep{goodfellow2014generative} to reconstruct images of stellar objects using II. We consider four Imaging Cherenkov Telescope Arrays (IACTs) located in the Northern Hemisphere and perform a one-night simulation for a fast-rotating star. The predicted image using a trained GAN shows promising results for reconstructing the image's shape and size. The evaluation of the predicted image is conducted using image moments, demonstrating that the size and shape of the fast-rotating star are reconstructed. However, the brightness distribution recovers only at the third-order moment.

This paper is organized as follows: The first section discusses Intensity Interferometry following its signal and noise for fast-rotating stars along the Earth's rotation. The second section introduces the GAN formulation and its structure. Following the third section details the parameter selection for training the GAN to reconstruct the image. After that, the fourth section presents the results of trained GAN visually and using image moments. At the end, there is a discussion of the entire result.