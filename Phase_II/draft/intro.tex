\section{Introduction}

Intensity Interferometry (II) \textcolor{blue}{was first reported by Hanbury Brown and Twiss (HBT) during the 1950s \citep{brown1954lxxiv, HBT56} as a \textquotedblleft new type of interferometry\textquotedblright \ to measure stellar parameters such as angular diameter, orbits, limb darkening coefficients etc. Later theoretical results reported by 
\cite{brown1957interferometry, brown1958interferometry}, 
and those of \cite{glauber1963quantum} and 
\cite{MandelWolf1995, Hecht2002} demonstrated the deeper 
physical properties of photon correlations that lay at the 
core of II and also laid the foundation of Quantum Optics.}
By the 1970s, \st{as a groundbreaking} \textcolor{blue}
{with the results of stellar parameter measurements of 32 stars 
in single and multistar systems, 
carried out by Hanbury Brown and his collaborators \citep{hanbury1974angular} at the historic Narrabri Stellar Intensity Interferometer (NSII), Australia,  
II had emerged as an alternative technique 
for measuring stellar parameters.}\st{diameters, orbits, and limb-darkening 
coefficients of 32 stars the N-body system.} Despite these significant achievements,
the method did not gain 
widespread adoption \textcolor{blue}{during several following decades due to unavailability of} the required 
sensitive photon detectors and sophisticated data analysis 
\textcolor{blue}{devices}\st{techniques}. 
However, \textcolor{blue}{subsequent} advancements in computational 
methods and in modern electronics, which currently offer time resolution of detectors in the nanoseconds (ns) to picoseconds (ps) range, have renewed 
II as a feasible technique for resolving astrophysical objects at visible wavelengths. Recent 
 simulations deploying ns-resolution photon detectors show that II 
 \st{is} \textcolor{blue}{could be} effective 
 in achieving high precision measurement of parameters for stellar objects 
 \citep{10.1093/mnras/stab2391, 10.1093/mnras/stac2433}. 

In addition to measurement of physical parameters of star systems, a fundamental motivation of optical astronomy is to obtain the sky intensity distribution of the source using the information observed on the ground. In the context of II, the task is to construct the image of the source from the intensity correlations observed by a set of pairs of telescopes (light buckets) on the ground.  \textcolor{blue}{However, with the basic observable of II being the electromagnetic field intensity, and not the field strength},
 a significant aspect of \textcolor{blue}{this method} is the loss of phase information. \st{Consequently, a} A full 
 reconstruction of the brightness distribution of a source with II 
 necessitates phase information. Therefoere the task is to reconstruct the phase of the signal. 

Several theoretical and computational approaches to phase reconstruction 
with II have been proposed. \cite{gamo1963triple} introduced the concept 
of triple-intensity correlation, which \cite{goldberger1963use} further 
applied in an experiment to observe scattered particles in microscopic systems. 
Sato conducted experiments to measure the diameter and phase of asymmetrical 
objects, suggesting that triple correlation could extend II to image stellar 
bodies \citep{sato1978imaging, sato1979computer, sato1981adaptive}. \textcolor{blue}{However, achieving a good signal-to-noise ratio (SNR) still poses a significant challenge for this approach.}

Gerchberg and Saxton suggested an iterative method \cite{GerchbergSaxton1972}to determine the phase of from the image and diffraction plane pictures. This method relies on good initial estimates and may struggle with slow convergence issues otherwise. Fienup suggested a Hybrid Input-Output algorithm that introduced feedback mechanisms to improve convergence rate and robustness, particularly in noisy environments.

Later \cite{holmes2010two} proposed an alternative method, utilizing the 
Cauchy-Riemann relations to reconstruct 1-D images and extending this to 
2-D images across a range of signal-to-noise (SNR) values. This algorithm 
was applied to simulated data of stellar objects using II, considering 
existing and forthcoming Imaging Cherenkov Telescope Arrays (IACTs) with 
a large number of telescopes 
\citep{nunez2010stellar, nunez2012high, nunez2012imaging}. This method faces challenge of computational complexity in attempting to generalize to higher dimensions. 

Li etal suggested a flexible iterative Regularization method \citep{Li2014} of incorporating prior information (e.g., sparsity, smoothness, or non-negativity), reducing the ill-posedness of the phase retrieval problem. This method is more robust against noise and stabilizes the solution against artefacts and spurious solutions. This method faces challenges of choice of regularization parameter, computational complexity and sensitivity to initial guess.  

The Transport-of-Intensity Equation (TIE) method, a non-interferometric technique and first suggested by Teague \citep{Teague1983} relates the intensity variations along the optical axis and phase of the optical fields. It enabled phase retrieval from intensity measurements at multiple planes. Zhang et al suggested a method \citep{Zhang2020} of obtaining a \textquotedblleft universal solution \textquotedblright to TIE using \textquotedblleft maximum intensity assumption \textquotedblleft, hence converting the TIE into a Poisson equation and then by solving it iteratively. Recent work of Kirisits et al \citep{Kirisits2024} have explored hybrid methods combining TIE with other equations, such as the Transport of Phase Equation (TPE). These approaches leverage the strengths of both equations to improve phase retrieval accuracy. This method has universal applicability in the sense that it works for arbitrarily shaped apertures, handles non-uniform illumination and inhomogeneous boundary conditions. This method guarantees convergence although the speed of convergence is dependent on initial guess quality. The final results are dependent on boundary conditions.

\textcolor{red}{With non-linearlity built into its grains, neural networks and deep learning were obvious choice to look for as an option to use. There have been these works.. involving deep learning, GAN.........{\bf Reference and brief description of other pieces of work using NN and Deep Learning to be inserted here!!}}

In this paper, we propose an alternative approach using Generative 
Adversarial Networks (GANs) \citep{goodfellow2014generative} to reconstruct 
images of stellar objects using II. We consider four Imaging Cherenkov 
Telescope Arrays (IACTs) located in the Northern Hemisphere and perform 
a one-night simulation for a fast-rotating star. The predicted image using 
a trained GAN shows promising results for reconstructing the image's shape and 
size.The evaluation of the predicted image is conducted using image moments, 
demonstrating that the size and shape of the fast-rotating star are 
reconstructed. However, the brightness distribution recovers only at the 
third-order moment.

This paper is organized as follows: The first section discusses 
Intensity Interferometry following its signal and noise for fast-rotating 
stars along the Earth's rotation. The second section introduces the GAN 
formulation and its structure. Following the third section details the 
parameter selection for training the GAN to reconstruct the image. 
After that, the fourth section presents the results of trained GAN visually 
and using image moments. At the end, there is a discussion of the entire result.