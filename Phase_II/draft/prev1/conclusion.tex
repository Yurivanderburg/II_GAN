\section{Conclusion}
Intensity Interferometry (II) is re-emerging as a promising technique to overcome the challenges associated with Amplitude Interferometry in the optical wavelength range. This resurgence is bolstered by advanced facilities at Imaging Cherenkov Telescope Arrays (ICTAs), which capture high-resolution images using large apertures and sensitive photon detectors capable of resolving signals on the nanosecond timescale \cite{dravins2013optical}. The Major Atmospheric Gamma Imaging Cherenkov Telescope (MAGIC), one of the world's leading CTAs, is now being employed for II, observing stellar objects with its 17-meter diameter mirror \cite{lorenz2004status}. MAGIC has already demonstrated the potential of II by measuring and comparing the diameters of individual stars \cite{abe2024performance}. Other existing arrays, such as the Very Energetic Radiation Imaging Telescope Array System (VERITAS) and the High Energy Stereoscopic System (HESS), are also advancing their II capabilities during periods when gamma-ray observations are not in progress \cite{kieda2021veritas, zmija2022optical}. Future observations, enabled by more advanced and sensitive instrumentation as well as the upcoming CTAO, promise to extend the boundaries of optical observations further.

However, II faces a significant drawback: it captures only the magnitude of the signal via photon correlation, resulting in a loss of phase information. This challenge of phase retrieval in II has been effectively addressed through machine learning techniques, particularly with conditional Generative Adversarial Networks (cGAN). Our study demonstrates that applying cGAN to II data successfully recovers the size, shape, and brightness distribution of a fast-rotating star. Evaluations based on image moments—specifically, the monopole, second, and third-order moments—support the effectiveness of cGAN in achieving accurate image reconstruction from a one-night simulation of II using six baselines. A critical factor in the reconstruction process is the extent of Fourier plane coverage, which depends on the number of available telescopes and the total observing time. With a full night of observation using four telescopes, the brightness distribution could be reconstructed with even greater precision. Future work might explore different observatory layouts to assess their impact on image reconstruction quality, such as integrating the Southern Cherenkov Telescope Array (CTA) with MAGIC.

While the results of this study highlight the significant potential of machine learning—particularly cGAN—for image reconstruction in II, several aspects require further refinement. First, detector efficiencies, which impact the signal-to-noise ratio (SNR) of actual observational data, have not yet been incorporated; addressing these factors will be crucial for more accurate SNR estimation. Second, exploring and comparing alternative methods for image generation could reveal approaches that outperform cGAN in reconstructing stellar images with II. Third, experimenting with different loss functions could provide additional insights into reconstruction quality. Although further testing is needed to refine the GAN and enhance its robustness and reliability, our findings suggest that machine learning is a promising approach for phase reconstruction in II.